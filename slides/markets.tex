% Created 2021-07-21 Mi 13:33
% Intended LaTeX compiler: pdflatex
\documentclass[bigger]{beamer}
\usepackage[utf8]{inputenc}
\usepackage[T1]{fontenc}
\usepackage{graphicx}
\usepackage{grffile}
\usepackage{longtable}
\usepackage{wrapfig}
\usepackage{rotating}
\usepackage[normalem]{ulem}
\usepackage{amsmath}
\usepackage{textcomp}
\usepackage{amssymb}
\usepackage{capt-of}
\usepackage{hyperref}
\mode<beamer>{\useinnertheme{rounded}\usecolortheme{rose}\usecolortheme{dolphin}\setbeamertemplate{navigation symbols}{}\setbeamertemplate{footline}[frame number]{}}
\mode<beamer>{\usepackage{amsmath}\usepackage{ae,aecompl,sgamevar}}
\let\oldframe\frame\renewcommand\frame[1][allowframebreaks]{\oldframe[#1]}
\setbeamertemplate{frametitle continuation}[from second]
\usetheme{default}
\author{Christoph Schottmüller}
\date{}
\title{Markets and the First Fundamental Welfare Theorem}
\hypersetup{
 pdfauthor={Christoph Schottmüller},
 pdftitle={Markets and the First Fundamental Welfare Theorem},
 pdfkeywords={},
 pdfsubject={},
 pdfcreator={Emacs 27.2 (Org mode 9.4.4)}, 
 pdflang={English}}
\begin{document}

\maketitle

\section{Exchange economy}
\label{sec:orga1465f8}
\begin{frame}[label={sec:org359d21d}]{Introduction}
\begin{itemize}
\item so far
\begin{itemize}
\item how to aggregate preferences
\item Arrow's impossibility theorem
\end{itemize}
\item today: a special aggregation problem
\begin{itemize}
\item exchange economy
\item similar to standard micro model in Bachelor
\item try to make the link:
\begin{itemize}
\item how is this a special case of the social choice model?
\item what additional structure/assumptions are in place?
\item which normative criteria do we use?
\item how do we avoid Arrow's impossibility theorem?
\end{itemize}
\end{itemize}
\end{itemize}
\end{frame}

\begin{frame}[label={sec:orgc03043a}]{A standard exchange economy}
\begin{itemize}
\item \(I\) consumers
\item \(n\) goods
\item consumer \(i\) has initial endowment \(e^i=(e_1^i,e_2^i,\dots,e_n^i)\) where \(e^i_j\in\Re_+\)
\begin{itemize}
\item assumption: each good exists in strictly positive quantities, \(\sum_{i=1}^Ie_j^i>0\) for all \(j=1,\dots,n\)
\end{itemize}
\item consumers preferences over consumption are represented by a utility function \(u^i:\Re_+^n\rightarrow\Re\)
\begin{itemize}
\item assumption: \(u^i\) is strictly increasing in each component
\item assumption: \(u^i\) is strictly quasi-concave
\item assumption: \(u^i\) is continuous
\end{itemize}
\item consumers can exchange endowments
\begin{itemize}
\item who should/will eventually consumer what?
\end{itemize}
\end{itemize}
\end{frame}
\begin{frame}[label={sec:orgf135851}]{Notation}
\begin{itemize}
\item \(e=(e^1,e^2,\dots, e^I)\) is the vector of endowments
\item allocations are denoted by \(x=(x^1,x^2,\dots ,x^I)\)
\begin{itemize}
\item \(x^i\in \Re^n_+\) is agent \(i\)'s allocation of the \(n\) good
\end{itemize}
\item feasible allocations:
$$F(e)=\{x|\sum_{i=1}^Ix^i=\sum_{i=1}^Ie^i\}$$
where each \(x^i\in\Re^n_+\)
\end{itemize}
\end{frame}

\begin{frame}[label={sec:org0058afe}]{Efficiency}
\begin{block}{Pareto efficiency}
An allocation \(x\in F(e)\) is Pareto efficient if there is no \(y\in F(e)\) such that \(u^i(y^i)\geq u^i(x^i)\) for all \(i=1,\dots,I\) with strict inequality for at least one \(i\).
\end{block}
\end{frame}


\begin{frame}[label={sec:org124977d}]{Comparison}
\begin{itemize}
\item does Arrow's impossibility theorem apply in this framework?
\end{itemize}
\end{frame}

\begin{frame}[label={sec:org2aab448}]{Prices and the consumer problem}
\begin{itemize}
\item \(p=(p_1,\dots,p_n)\) be a vector of prices (\(p_j\) is the price of good  \(j\)) and assume \(p_j>0\) for all \(j=1,\dots,n\)
\item assumption: each consumer takes the vector of prices as given
\item consumer \(i\)'s problem:
$$\max_{x^i\in\Re^n_+}u^i(x^i)\qquad s.t.:\quad \sum_{j=1}^np_jx_j^i\leq  \sum_{j=1}^np_je_j^i$$
\item think of \(m^i(p)=\sum_{j=1}^np_je_j^i\) as consumer \(i\)'s income
\item given our assumptions, what can we say about the solution of this problem?
\end{itemize}
\end{frame}
\begin{frame}[label={sec:org12be0b3}]{Solution to the consumer problem}
\begin{itemize}
\item solution exists
\item solution is unique
\item denote the solution to the consumer problem as \(x^i(p,m^i(p))\)
\item \(x^i(p,m^i(p))\) is continuous
\item demand is homogenous: \(x^i(p,m^i(p))=x^i(\lambda p,m^i(\lambda p))\)
\item budget constraint holds with equality
\item the marginal rate of substitution between any two goods equals the price ratio
$$MRS_{j,k}^i=-\frac{\partial u^i/\partial x^i_j}{\partial u^i/\partial x^i_k}= -\frac{p_j}{p_k}$$
\end{itemize}
\end{frame}

\begin{frame}[label={sec:org87d3ee0}]{Excess demand}
\begin{itemize}
\item aggregate excess demand for good \(j\) is defined as
$$z_j(p)=\sum_{i=1}^Ix_j^i(p,m^i(p))-\sum_{i=1}^Ie^i_j$$
\begin{itemize}
\item if \(z_j(p)>0\) demand for good \(j\) is higher than its supply at price \(p\)
\item if \(z_j(p)<0\) demand for good \(j\) is lower than its supply at price \(p\)
\end{itemize}
\item aggregate excess demand is defined as
$$z(p)=(z_1(p),z_2(p),\dots,z_n(p))$$
\end{itemize}
\end{frame}

\begin{frame}[label={sec:orgb9745b5}]{Properties of excess demand}
\begin{block}{Proposition}
Under our assumptions, excess demand satisfies
\begin{itemize}
\item continuity: \(z\) is continuous at \(p\)
\item homogeneity: \(z(\lambda p)=p\) for all \(\lambda\in\Re_{++}\)
\item Walras' law: \(\sum_{j=1}^n p_j z_j(p)=0\)
\end{itemize}
\end{block}
\begin{block}{Proof}
\begin{itemize}
\item continuity:
\item homogeneity:
\item Walras law:
\(\sum_{j=1}^n p_j z_j(p)=\sum_{j=1}^n p_j \left( \sum_{i=1}^Ix_j^i(p,m^i(p))-\sum_{i=1}^Ie^i_j  \right)\)
\(= \sum_{i=1}^I \sum_{j=1}^n\left( p_j x_j^i(p,m^i(p))-p_je^i_j\right)\)
\(= \sum_{i=1}^I\left[ \sum_{j=1}^n\left( p_j x_j^i(p,m^i(p))\right)-m^i(p) \right]=0\) as budget constraint of each consumer holds with equality
\end{itemize}
\end{block}
\end{frame}


\begin{frame}[label={sec:org4d7576d}]{Implications of Walras' law}
\begin{itemize}
\item suppose we have only 2 goods (\(n=2\)) and we have at price vector \(p\) excess demand in market 1, \(z_1(p)<0\)
\begin{itemize}
\item what can we say about market 2?
\end{itemize}
\item let \(n>2\), if we have excess demand in good 1, \(z_1(p)<0\), what can we say about other markets?
\item if \(n-1\) markets are have zero excess demand, i.e. \(z_j(p)=0\) for \(j=1,\dots,n-1\), what can we say about the remaining market?
\end{itemize}
\end{frame}

\begin{frame}[label={sec:orgcf10c9d}]{Walrasian equilibrium}
\begin{block}{Definition: Walrasian equilibrium}
A vector \(p^*\in\Re^n_{++}\) is called a Walrasian equilibrium if \(z(p^*)=0\).
\end{block}
\begin{itemize}
\item all market demands connected
\item "general equilibrium"
\end{itemize}
\end{frame}

\begin{frame}[label={sec:org612c21b}]{Walrasian equilibrium: Existence}
\begin{block}{Existence theorem}
A Walrasian equilibrium \(p^*\) exists. 
\end{block}
\begin{block}{Proof existence theorem}
somewhat technical, see Jehle and Reny (2011), ch. 5.2.1
\end{block}
\end{frame}
\begin{frame}[label={sec:org9d92ea0}]{Walrasian equilibrium: Efficiency}
\begin{block}{First fundamental theorem of welfare economics}
Let \(p^*\) be a Walrasian equilibrium. The equilibrium allocation \(x^*=(x^1(p^*),x^2(p^*),\dots,x^I(p^* ))\) is Pareto efficient.
\end{block}
\end{frame}
\begin{frame}[label={sec:orgb5f8719}]{Proof of the first fundamental theorem of welfare economics:}
\begin{itemize}
\item Suppose, to the contrary, that \(y=(y^1,\dots,y^I)\) Pareto dominates \(x^*\).
\begin{itemize}
\item Then, \(\sum_{j=1}^n p^*_j y_j^i\geq m^i(p^*)\) for all \(i\) with strict inequality for at least one \(i\) (Why?)
\vspace*{0.3cm}

$$\Rightarrow\sum_{i=1}^I\sum_{j=1}^n p^*_j y_j^i>\sum_{i=1}^I \sum_{j=1}^n p^*_je^i_j$$
\item \(y\) must be feasible:$$\sum_{i=1}^I y^i\leq\sum_{i=1}^I e^i$$  (note: there are vectors on both sides of the inequality!)
\begin{itemize}
\item hence, \(p^* \cdot \sum_{i=1}^I y^i\leq p^*\cdot\sum_{i=1}^I e^i\) as all \(p^*_j>0\) \linebreak (note: this is a dot/vector product)
\end{itemize}
$$\Rightarrow\sum_{i=1}^I\sum_{j=1}^n p^*_j y_j^i\leq \sum_{i=1}^I \sum_{j=1}^n p^*_je^i_j$$
\end{itemize}
\end{itemize}
\end{frame}


\begin{frame}[label={sec:org6a6a6f8}]{Example: 2 agents, 2 goods (Edgeworth box)}
\end{frame}

\begin{frame}[label={sec:orgd538388}]{First fundamental  theorem of welfare economics: comments}
\begin{itemize}
\item market system leads to efficient allocation
\item there are more general versions of this theorem
\begin{itemize}
\item with production, weaker assumptions on consumer preferences, etc.
\end{itemize}
\item decentralized market mechanisms can lead to efficient outcome
\begin{itemize}
\item or: a centralized solution can be implemented in a decentralized way using only prices
\end{itemize}
\end{itemize}
\end{frame}


\begin{frame}[label={sec:orgc083651}]{Aside: the role of prices I}
\begin{itemize}
\item the economic problem (putting all resources to their best use) is Herculean at society level
\begin{itemize}
\item what is best use?\linebreak \(\rightarrow\) requires knowledge of preferences
\item what are resources? \linebreak \(\rightarrow\) requires knowledge of
\begin{itemize}
\item possible production processes
\item natural resources
\item local conditions
\item possible labor supply and preferences concerning labor supply
\item transportation (im-)possibilities
\item \ldots{}
\end{itemize}
\end{itemize}
\end{itemize}
\end{frame}
\begin{frame}[label={sec:org13b93be}]{Aside: the role of prices II}
\begin{itemize}
\item planning problem becomes a problem of how to aggregate dispersed information
\begin{itemize}
\item unrealistic to centralize all this information
\item decentralized solution
\begin{itemize}
\item decisions should be made by those that most naturally have most of the necessary information
\item still need enough knowledge of outside world
\end{itemize}
\end{itemize}
\item prices aggregate all the information a decision maker needs to make the best decision for society
\begin{itemize}
\item consumer knows his own preferences
\item Walrasian price captures opportunity benefit of the resource, i.e. the value of the resource to others
\item each agent can act in interest of society without having to know/understand the interest of society
\item what does an increasing price signal?
\end{itemize}
\item do you know the famous  \alert{\href{https://youtu.be/jPbh4NyKH0M}{pencil clip}}?
\end{itemize}
\end{frame}

\begin{frame}[label={sec:orgdeff8d5}]{First fundamental  theorem of welfare economics: important assumptions}
\begin{itemize}
\item all agents are price takers
\item complete markets
\begin{itemize}
\item every good that matters for some consumer is traded on its own market
\item guaranteed property rights, i.e. voluntary trade is possible (no theft etc.)
\end{itemize}
\item note:
\begin{itemize}
\item assumptions are sufficient to reach efficiency
\item an efficient equilibrium may still exist if some of the assumptions fail!
\end{itemize}
\end{itemize}
\end{frame}


\begin{frame}[label={sec:orgb1576ee}]{Violations of assumptions}
\begin{itemize}
\item agents are price takers
\begin{itemize}
\item examples of cases where agents are not price takers?
\vspace*{1cm}
\end{itemize}
\item complete markets assumption
\begin{itemize}
\item a good is not traded on a market:
\vspace*{1cm}
\item distinct goods are traded on a common market:
\vspace*{1cm}
\end{itemize}
\end{itemize}
\end{frame}
\begin{frame}[label={sec:orgb8a0ae6}]{The scope for policy: efficiency arguments}
\begin{itemize}
\item policy within model:
\begin{itemize}
\item guarantee property rights + enforce contracts
\end{itemize}
\item Efficiency reached without policy intervention given our assumptions.
\item failure of assumptions is necessary but not sufficient for existence of efficiency enhancing policy
\begin{itemize}
\item outcome may still be efficient
\item efficiency enhancing policy may not be available
\end{itemize}
\item reactions if assumptions fail that are motivated by model
\begin{itemize}
\item competition policy and sector regulation
\item complete/create the market
\end{itemize}
\end{itemize}
\end{frame}

\begin{frame}[label={sec:orga1c95ac}]{Aside: The scope for policy: distributional arguments}
\begin{itemize}
\item second fundamental theorem of welfare economics:\linebreak
any efficient allocation is a Walrasian equilibrium for some vector of endowments
\item implication
\begin{itemize}
\item realize distributional objectives by redistributing endowments only
\item then let market ensure efficiency
\end{itemize}
\item some caveats to this
\end{itemize}
\end{frame}
\begin{frame}[label={sec:org2a74418}]{Walrasian equilibrium: how to get there?}
\begin{itemize}
\item how do markets reach a Walrasian equilibrium?
\item how do we obtain prices if everyone is price taker?
\item metaphor of Walrasian auctioneer
\vspace*{1cm}
\item maybe a good idea to talk about the economics of auctions
\begin{itemize}
\item for auction theory, we need game theory with incomplete information
\item for game theory with incomplete information we need decision making under uncertainty
\end{itemize}
\ldots{}that's exactly the plan for the coming weeks!
\end{itemize}
\end{frame}
\end{document}