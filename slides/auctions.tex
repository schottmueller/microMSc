% Created 2021-08-20 Fr 16:04
% Intended LaTeX compiler: pdflatex
\documentclass[bigger]{beamer}
\usepackage[utf8]{inputenc}
\usepackage[T1]{fontenc}
\usepackage{graphicx}
\usepackage{grffile}
\usepackage{longtable}
\usepackage{wrapfig}
\usepackage{rotating}
\usepackage[normalem]{ulem}
\usepackage{amsmath}
\usepackage{textcomp}
\usepackage{amssymb}
\usepackage{capt-of}
\usepackage{hyperref}
\mode<beamer>{\useinnertheme{rounded}\usecolortheme{rose}\usecolortheme{dolphin}\setbeamertemplate{navigation symbols}{}\setbeamertemplate{footline}[frame number]{}}
\mode<beamer>{\usepackage{amsmath}\usepackage{ae,aecompl,sgamevar,tikz}}
\let\oldframe\frame\renewcommand\frame[1][allowframebreaks]{\oldframe[#1]}
\setbeamertemplate{frametitle continuation}[from second]
\newcommand{\Ra}{\Rightarrow} \newcommand{\ra}{\rightarrow} \newcommand{\Lra}{\Leftrightarrow}
\usetheme{default}
\author{Christoph Schottmüller\thanks{I want to thank Ole Jann (CERGE EI) for giving me access to his lecture slides. Many of the following slides are based on his material.}}
\date{}
\title{Auctions}
\hypersetup{
 pdfauthor={Christoph Schottmüller\thanks{I want to thank Ole Jann (CERGE EI) for giving me access to his lecture slides. Many of the following slides are based on his material.}},
 pdftitle={Auctions},
 pdfkeywords={},
 pdfsubject={},
 pdfcreator={Emacs 27.2 (Org mode 9.4.4)}, 
 pdflang={English}}
\begin{document}

\maketitle

\section{Introduction to auctions}
\label{sec:org938fe33}

\begin{frame}[label={sec:orge66a8d1}]{Auctions are used in many contexts I}
\begin{itemize}
\item privatization of companies, drilling rights, mining rights, mobile phone spectrum, \ldots{}
\item art, furniture, fresh flowers, fish, houses,\ldots{}
\item treasury bills, bonds, other debts,\ldots{} \linebreak \(\Rightarrow\) billions of dollars every week!
\item public and private procurement (reverse auction: lowest price wins)
\item historical examples: auction of the Roman empire in 193 AD, bidding for wives in Babylonia\footnote{\tiny Though famous, it is controversial whether Herodotus' account is historically accurate, see, for instance, \url{https://www.jstor.org/stable/4436038}.},\ldots{}
\item Google (and others) ad auctions (\(\approx\) 150 billion \(\$\) in 2020)
\item electricity wholesale markets
\end{itemize}
\end{frame}
\begin{frame}[label={sec:orgc0687a6}]{Auctions are used in many contexts II}
\begin{itemize}
\item many other situations can be thought of as auctions:
\begin{itemize}
\item take-over battles for a company
\item queuing for something, lobbying politicians, advertising (“all-pay auction” or “war of attrition”)
\item markets\ldots{}
\end{itemize}
\end{itemize}
\end{frame}

\begin{frame}[label={sec:orgb8892c9}]{Auction design matters}
\begin{itemize}
\item most visible application of auction theory in recent decades: Mobile spectrum auctions
\item starting in the late 90s, all developed countries sold off mobile spectrum to companies to be used for mobile internet
\item countries differed in size, number of incumbent companies, number of available licenses \ldots{}\linebreak \(\Rightarrow\)  all used slightly different auction formats
\item per capita revenue 3G spectrum auctions:
\begin{itemize}
\item €650 in the UK
\item €615 in Germany
\item €170 in the Netherlands
\item €20 in Switzerland
\end{itemize}
\item “economic theorists deserve some of the blame” (Klemperer 2003)
\end{itemize}
\end{frame}

\begin{frame}[label={sec:org66791c9}]{Auctions: A theory and a tool}
\begin{itemize}
\item auctions are both a theoretical concept and a practical tool
\begin{itemize}
\item thinking about their theoretical properties has larger benefits for our understanding of economics \ldots{}
\item \ldots{} but it also helps inform their practical use
\item practical tool that has been used for millennia to allocate goods
\item thinking about their practical use requires theory, but also detailed knowledge and “street smarts”
\end{itemize}
\item our plan:
\begin{enumerate}
\item What are the most common auction formats? Which one maximizes revenue?
\item What are the advantages and disadvantages of different auction formats?
\end{enumerate}
\end{itemize}
\end{frame}

\section{Auction formats}
\label{sec:org2b4d17e}
\begin{frame}[label={sec:orge7ff8d6}]{Private values framework}
\begin{itemize}
\item \(I\) bidders and one object to be sold
\item bidder \(i\) has valuation \(v_i\) for the object
\item \(v_i\) is private information of bidder \(i\) (his \emph{type})
\item all \(v_i\) are identically and independently distributed on \([\underline v,\bar v]\) with cdf \(F\)
\begin{itemize}
\item for simplicity: uniform distribution on \([0,1]\)
\end{itemize}
\item bidder \(i\)'s payoff is
\begin{itemize}
\item valuation \(v_i\) minus payment if he gets object
\item minus payment (if any) if he does not get the object
\end{itemize}
\item we denote \(i\)'s bid as \(b_i\)
\begin{itemize}
\item \(i\)'s bidding strategy is a function \(b_i:\,[\underline v,\bar v]\ra \Re_+\)
\end{itemize}
\end{itemize}
\end{frame}

\begin{frame}[label={sec:orgf1fe4b9}]{Common auction formats for a single object}
\begin{itemize}
\item sealed-bid auctions:
\begin{itemize}
\item first-price sealed-bid: Every bidder submits their offer, the highest bidder wins and pays his bid
\item second-price sealed-bid: Every bidder submits their offer, the highest bidder wins and pays the second-highest bidder’s bid
\end{itemize}
\item Open auctions:
\begin{itemize}
\item descending ("Dutch"): The auctioneer announces a high price and then slowly decreases it, until a bidder says that he wants to buy at that price
\item ascending ("English"): The auctioneer starts with a low price and announces higher prices, until only one bidder remains and pays the last announced price
\end{itemize}
\end{itemize}
Which one will bring the highest revenue? \linebreak We need to understand how bidders will behave!
\end{frame}

\begin{frame}[label={sec:org94b85ab}]{First-price sealed-bid auction (FPA)}
\begin{itemize}
\item Every bidder submits their offer, the highest bidder wins and pays his bid.
\item If your valuation is \(v_i\) , should you bid \(v_i\) ?
\pause
\item basic trade-off for any bidder:
\begin{itemize}
\item higher bid increases chance of winning\ldots{}
\item \ldots{}but decreases surplus in case of winning
\end{itemize}
\item \(\Ra\) ”bid shading”: In equilibrium, every bid \(b_i\) will be below bidder’s value \(v_i\).
\end{itemize}
\end{frame}

\begin{frame}[label={sec:org93b0686}]{Descending (“Dutch”) auction}
\begin{itemize}
\item The auctioneer announces a high price and then slowly decreases it, until a bidder says that he wants to buy at that price.
\item Once announced price p is below \(v_i\), there is a basic trade-off:
\begin{itemize}
\item agreeing to pay earlier increases chance of winning (i.e. nobody else agrees before)\ldots{}
\item \ldots{}but decreases surplus in case of winning
\end{itemize}
\item \(\Ra\) ”Bid shading”
\item FPA and Dutch auction require the same considerations
\item FPA and Dutch auction are actually strategically equivalent:
\begin{itemize}
\item same strategy set \(S_i=\Re_+\)
\item same payoff \(u_i(s)\) for every given strategy profile \(s\)
\item same Bayesian game!
\end{itemize}
\end{itemize}
\end{frame}

\begin{frame}[label={sec:org0ba46aa}]{Second-price sealed-bid auction (SPA)}
\begin{itemize}
\item Every bidder submits their offer, the highest bidder wins and pays the second-highest bidder’s bid.
\item With value \(v_i\) , should you ever submit bid \(b_i > v_i\) instead of \(b_i = v_i\) ?
\begin{itemize}
\item only changes the outcome if second-highest bid \(b_j\) is between \(v_i\) and \(b_i\)
\begin{itemize}
\item if \(b_j > b_i\) you don’t win either way
\item if \(b_j < v_i\) you win and pay \(b_j\) either way
\end{itemize}
\item But then you win and pay \(b_j > v_i\) and make negative surplus \(\Ra\)  better to bid \(b_i = v_i\)
\pause
\end{itemize}
\item Should you ever bid \(b_i < v_i\) instead of \(b_i = v_i\) ?
\begin{itemize}
\item only changes the outcome if \(b_j \in (b_i , v_i )\)
\begin{itemize}
\item if \(b_j > v_i\) you don’t win either way
\item if \(b_j < b_i\) you win and pay \(b_j\) either way
\end{itemize}
\item But then you lose, whereas with \(b_i = v_i\) you would have won and made positive surplus \(\Ra\) better to bid \(b_i = v_i\)
\end{itemize}
\item \(\Ra\) It is a weakly dominant strategy to bid \(b_i = v_i\)
\end{itemize}
\end{frame}

\begin{frame}[label={sec:org16cc35f}]{Ascending (“English”) auction}
\begin{itemize}
\item The auctioneer starts with a low price and announces higher prices, until only one bidder remains and pays the last announced price.
\item Should you ever leave the bidding while announced price \(p\) is below \(v_i\) ?
\begin{itemize}
\item No: stay and either win at some \(p < v_i\) (positive surplus), or leave at \(v_i\) (zero surplus)
\pause
\end{itemize}
\item Should you ever stay in the bidding when announced price \(p\) is above \(v_i\) ?
\begin{itemize}
\item No: leaving guarantees zero surplus while staying in might lead to negative surplus (and never to positive)
\end{itemize}
\item \(\Ra\) weakly dominant strategy to stay in the auction until \(p = v_i\)
\item (with independent values) SPA and English auction are strategically equivalent
\end{itemize}
\end{frame}

\begin{frame}[label={sec:org28b4086}]{Optimal bidder behavior in the first-price auction I}
\begin{itemize}
\item What is the equilibrium strategy \(b_i:[\underline v,\bar v]\ra\Re_+\) in a FPA (or in a Dutch) auction?
\item We will consider a simple case:
\begin{itemize}
\item two bidders have with \(v_i\) independently and uniformly distributed on \([0, 1]\)
\end{itemize}
\item Assume that there exists a symmetric bidding equilibrium where everybody follows the increasing bidding strategy \(\beta: [\underline v,\bar v]\ra\Re_+\).
\item If \(j\) follows strategy \(\beta\), \(i\)'s expected payoff from bidding \(b_i\) is
\end{itemize}
\begin{multline*}
Pr[b_i>\beta(v_j)]*(v_i-b_i) \\+ Pr[b_i=\beta(v_j)]\frac{1}{2}*(v_i-b_i)+Pr[b_i<\beta(v_j)]*0
\end{multline*}
\end{frame}

\begin{frame}[label={sec:orgdf3f7b5}]{Optimal bidder behavior in the first-price auction II}
\begin{itemize}
\item If \(j\) follows strategy \(\beta\), \(i\)'s expected payoff from bidding \(b_i\) is
\end{itemize}
\begin{multline*}Pr[b_i>\beta(v_j)]*(v_i-b_i)= Pr[v_j<\beta^{-1}(b_i)]*(v_i-b_i)\\
= \beta^{-1}(b_i)*(v_i-b_i)
\end{multline*}
\vspace*{-0.5cm}
\begin{itemize}
\item first order condition of maximizing payoff over bid \(b_i\)
\end{itemize}
$$ {\beta^{-1}}'(b_i)*(v_i-b_i)-\beta^{-1}(b_i)=0$$
$$\Lra \frac{1}{\beta'(\beta^{-1}(b_i))}*(v_i-b_i)-\beta^{-1}(b_i)=0$$
\begin{itemize}
\item if \(\beta\) is equilibrium, maximum is achieved at \(b_i=\beta(v_i)\) and therefore
$$\frac{v_i-\beta(v_i)}{\beta'(v_i)}-v_i=0$$
\item this differential equation is solved by $$\beta(v_i)=v_i/2$$
\end{itemize}
\end{frame}

\begin{frame}[label={sec:org83df89c}]{Optimal bidder behavior in the first-price auction III}
\begin{itemize}
\item both players bidding according to the strategy \(\beta(v_i)=v_i/2\) is equilibrium!
\begin{itemize}
\item derivation maybe tricky but\ldots{}
\item \ldots{}make sure you understand that bidding \(v_i/2\) is type \(v_i\)'s best response if \(j\) uses the strategy \(b_j(v_j)=v_j/2\)
\end{itemize}
\item a lot of bid shading in equilibrium!
\begin{itemize}
\item with \(I\) players (and iid uniformly distributed values) the equilibrium is
\end{itemize}
$$\beta(v_i)=\frac{I-1}{I}v_i$$
\begin{itemize}
\item \(\Ra\) bid shading decreases with number of bidders
\end{itemize}
\end{itemize}
\end{frame}

\begin{frame}[label={sec:org5060006}]{Which auction yields higher revenue?}
\begin{itemize}
\item For \(v_1 , v_2\) uniformly iid on \([0,1]\), which auction gives the higher revenue?
\item revenue of the FPA
\begin{itemize}
\item expected revenue from bidder \(i\):
$$\int_0^1 Pr[v_j<v_i]\frac{v_i}{2}\,dv_i=\int_0^1 v_i\frac{v_i}{2}\,dv_i=\frac{1}{6} $$
\item total expected revenue: \(2*1/6=1/3\)
\end{itemize}
\item revenue of the SPA
\begin{itemize}
\item expected revenue from bidder \(i\):
$$\int_0^1 Pr[v_j<v_i]\mathbb{E}[v_j|v_j<v_i]\,dv_i=\int_0^1 v_i \frac{v_i}{2}\,dv_i=\frac{1}{6} $$
\item total expected revenue: \(2*1/6=1/3\)
\end{itemize}
\end{itemize}
\end{frame}

\begin{frame}[label={sec:org50fabf2}]{Conclusions so far}
\begin{itemize}
\item with independent private values
\begin{itemize}
\item first price sealed bid and Dutch auction are strategically equivalent
\item second price sealed bid and ascending auction are strategically equivalent
\end{itemize}
\item with independent private values and uniformly distributed types
\begin{itemize}
\item both auctions are efficient (bidder with highest valuation gets the good)
\item both auctions yield the same expected revenue
\end{itemize}
\end{itemize}
\end{frame}

\begin{frame}[label={sec:org02405fc}]{Aside: Information rents and commitment}
\begin{itemize}
\item for a moment, suppose the seller knows the bidders' valuations
\begin{itemize}
\item how will he maximize revenue?
\item what is the expected utility of a buyer?
\end{itemize}
\item how does this compare to expected buyer utility when the seller does not observe buyer valuations?
\pause
\item private information creates an "information rent"
\item the seller can infer valuations from bids in auction
\begin{itemize}
\item incentive to cancel auction and switch to procedure above\ldots{}
\item \ldots{}but buyers anticipating would then not participate in auction (or bid lower)
\item importance of \emph{commitment}
\begin{itemize}
\item seller will not change auction rules midway
\end{itemize}
\end{itemize}
\end{itemize}
\end{frame}


\section{Revenue equivalence}
\label{sec:orgf103267}
\begin{frame}[label={sec:org7c3703d}]{Revenue equivalence theorem (RET)}
\begin{block}{Revenue equivalence theorem}
Suppose that valuations are identically and independently distributed with strictly positive density on \([\underline v,\bar v]\) and that bidders are risk neutral.\linebreak
Then, every \emph{(auction format, equilibrium)} pair such that in equilibrium
\begin{itemize}
\item the object is won by the bidder with the highest valuation and
\item a bidder of type \(\underline v\) has zero expected utility
\end{itemize}
gives the same revenue to the seller and the same expected utility to all types.
\end{block}

\begin{itemize}
\item revenue equivalence is much more general (regardless of distribution, auction type etc.)
\end{itemize}
\end{frame}

\begin{frame}[label={sec:org38affd5}]{Revenue equivalence theorem: theoretical relevance}
\begin{itemize}
\item RET as a theoretical tool
\begin{itemize}
\item can help us think through changes in auction format and whether they will have an effect or not
\item helps us understand complicated auctions by relating strategies and outcomes to simpler auctions
\item can be used to derive equilibria in a simple manner (example for this later on)
\item is a starting point for further analysis
\begin{itemize}
\item how does the comparison of revenue change if assumptions of RET fail?
\end{itemize}
\end{itemize}
\end{itemize}
\end{frame}

\begin{frame}[label={sec:org94fb9a8}]{Revenue equivalence theorem: economic relevance}
\begin{itemize}
\item RET as an economic insight
\begin{itemize}
\item Does it mean that in real life, all auction formats will have the same revenue? No!
\item But there is no a priori reason to expect that any auction format would always be better than others \ldots{}
\item \ldots{} or that any format would be better for some people than another format.
\item One weak formulation with real-life relevance:
\begin{itemize}
\item If we change the rule to increase expected payment for a given bid \(b_i\), bidders will lower their bids \(b_i\) to compensate for that
\item \(\Ra\) Under idealized conditions, these effects exactly cancel each other out
\end{itemize}
\end{itemize}
\end{itemize}
\end{frame}

\begin{frame}[label={sec:orgf2db9cf}]{Revenue equivalence theorem: proof I}
\begin{itemize}
\item take an (equilibrium, auction format) pair and assume that assumptions of RET are satisfied
\item define \(U_i(v_i)=v_i P(v_i)- T_i(v_i)\) where
\begin{itemize}
\item \(U_i\) is expected utility of player \(i\) in equilibrium
\item \(P\) is probability that all other bidders have a lower type than \(v_i\), i.e. \(P(v_i)=F(v_i)^{I-1}\)
\item \(T_i\) is the expected amount of money \(i\) pays to the auctioneer in equilibrium if he has type \(v_i\)
\end{itemize}
\end{itemize}
\end{frame}
\begin{frame}[label={sec:orgabb4e34}]{Revenue equivalence theorem: proof II}
\begin{itemize}
\item define \(U_i(v_i)=v_i P(v_i)- T_i(v_i)\)
\item \alert{envelope theorem:} \(U_i'(v_i)=P(v_i)\)
\begin{itemize}
\item in equilibrium type \(v_i'\) prefers his bid to the bid of type \(v_i''\)
$$U_i(v_i')\geq v_i' P(v_i'') -T_i(v_i'')=U_i(v_i'')+(v'-v'')P(v_i'')$$
\item in equilibrium type \(v_i''\) prefers his bid to the bid of type \(v_i'\)
$$U_i(v_i'')\geq v_i'' P(v_i') -T_i(v_i')=U_i(v_i')-(v'-v'')P(v_i')$$
\item taking these two inequalities together (and let \(v_i'>v_i''\))
$$P(v_i')\geq \frac{U_i(v_i')-U_i(v_i'')}{v_i'-v_i''}\geq P(v_i'')$$
\item taking limit \(v_i''\rightarrow v_i'\) gives (as \(P\) is continuous)
$$U_i'(v_i')=P(v_i)$$
\end{itemize}
\end{itemize}
\end{frame}
\begin{frame}[label={sec:orgafbb00b}]{Revenue equivalence theorem: proof III}
\begin{itemize}
\item envelope theorem: \(U_i'(v_i)=P(v_i)\)
\item hence,
\begin{multline*}U_i(v_i)=U_i(\underline v) + U_i(v_i)-U_i(\underline v)=U_i(\underline v)+\int_{\underline v}^{v_i}U_i'(v)\,dv\\=U_i(\underline v)+\int_{\underline v}^{v_i}P(v)\,dv=\int_{\underline v}^{v_i}P(v)\,dv= \int_{\underline v}^{v_i}F(v)^{I-1}\,dv\end{multline*}
\item expected utility of player \(i\) of type \(v_i\) does not depend on specific auction format or equilibrium!
\item same is true for expected payment of type \(v_i\) of player \(i\):
$$T_i(v_i)=v_i P(v_i) -U_i(v_i)=v_i F(v_i)^{I-1} -\int_{\underline v}^{v_i}F(v)^{I-1}\,dv$$
\item expected revenue is just \(\sum_{i=1}^I \mathbb{E}[T_i(v_i)]\) which therefore also does not depend on auction format or equilibrium\qed
\end{itemize}
\end{frame}

\begin{frame}[label={sec:org4d3d658}]{Revenue equivalence theorem: proof idea}
\begin{itemize}
\item the envelope theorem states that the derivative of bidder utility only depends on the type distribution (given that the highest type wins)
\item assumption that the lowest type has zero utility
\item together this implies that the utility of every type depends only on the type distribution (and not the auction format, equilibrium strategies etc.)
\item bidder utility only depends on the type distribution!
\item welfare (defined as expected valuation of the person who get the good) is the same in all auctions as the highest bidder always gets the good
\item revenue is difference between welfare and bidder utility\ldots{}done!
\item note: envelope theorem does all the real work!
\end{itemize}
\end{frame}

\begin{frame}[label={sec:org94f5caf}]{Applying RET I}
\begin{itemize}
\item Consider the following auction formats:
\begin{itemize}
\item English (i.e. ascending) auction
\item The highest bidder gets the object and pays twice her bid
\item The highest bidder gets the object and pays the sum of her bid and the next-highest bid
\item An ascending auction is held until only two bidders remain. Then these two are asked to submit sealed bids; the highest bidder gets the object and pays her bid (“Anglo-Dutch Auction”)
\end{itemize}
\item Which one will give the highest expected revenue?
\pause
\begin{itemize}
\item It seems reasonable that there are symmetric equilibria in strictly increasing strategies.
\item They will all give the same expected revenue (and in expectation make the bidders equally well off)!
\end{itemize}
\end{itemize}
\end{frame}

\begin{frame}[label={sec:org203af10}]{Applying RET IIa}
\begin{itemize}
\item 2 companies are engaged in a legal battle
\item each company chooses how much money to spend on lawyers
\item company that invests the more wins the battle and gets a value of \(v_i\) , where \(v_i\) are iid and uniform on \([0,1]\)
\item This is an all-pay auction: Every company chooses a bid (= how much to pay for lawyers) and pays this for sure, but only the one who pays more wins the prize.
\item How will each company spend on lawyers in equilibrium, if it only knows its own valuation \(v_i\)?
\item Calculating this directly would be slightly more complicated than solving a FPA \ldots{}
\item \ldots{}but the RET can help us!
\end{itemize}
\end{frame}

\begin{frame}[label={sec:org6175575}]{Applying RET IIb}
\begin{itemize}
\item consider the corresponding SPA: equilibrium to bid own valuation
\item if \(i\) wins with bid \(b_i=v_i\), the expected bid of \(j\) is \(\mathbb{E}[v_j|v_j<v_i]=v_i/2\)
\item expected payment of type \(v_i\) in SPA is probability of winning, i.e. \(v_i\), times expected payment when winning
$$T_i(v_i)=\mathbb{E}[v_j|v_j<v_i]Pr[v_j<v_i]=v_i/2 * v_i=v_i^2/2$$
\item assume the all pay auction has a symmetric equilibrium in which the highest type wins
\begin{itemize}
\item wining probability same as in SPA for every type
\item expected payment must be the same for every type as in SPA by RET
\item equilibrium bid in all pay auction must be \(v_i^2/2\)! done!
\end{itemize}
\end{itemize}
\end{frame}

\section{Choosing the right auction format}
\label{sec:orgbd5f19a}

\begin{frame}[label={sec:org519cb39}]{Challenges in practical auction design}
\begin{itemize}
\item theoretical:
\begin{itemize}
\item understanding and solving the auction
\end{itemize}
\item competition policy:
\begin{itemize}
\item collusion among bidders
\item enough bidders must enter
\item predatory behavior by powerful/rich bidders
\end{itemize}
\item behavioral:
\begin{itemize}
\item bidding is done by actual humans (or companies) with image concerns etc.
\item solving for equilibria can be hard (cognitive constraints)
\item risk-aversion, or other non-standard utility functions
\end{itemize}
\item context/institutional:
\begin{itemize}
\item credibility and commitment of auctioneer
\item political pressures on auction designer
\end{itemize}
\item We will today discuss some examples of the above.
\end{itemize}
\end{frame}

\begin{frame}[label={sec:orgfa6f882}]{Collusion in SPA}
\begin{itemize}
\item consider a second-price auction
\item If bidders can figure out that \(i\) has the highest valuation, the
\end{itemize}
following is an equilibrium:
\begin{itemize}
\item i bids \(v_i\) , everybody else bids 0
\item \(\Ra\) \(i\) gets the object at price 0
\end{itemize}
\begin{itemize}
\item Bidders can accomplish this by forming a bidding ring and running a pre-auction knock-out.
\end{itemize}
\end{frame}

\begin{frame}[label={sec:org9aacaf4}]{Pre-auction knockout I}
\begin{itemize}
\item Consider the following mechanism:
\begin{enumerate}
\item ring organizer asks each member to report their valuation \(v_i\)
\item member with the highest announced valuation \(\hat v_i\) “represents” the ring and submits his actual bid; everybody else drops out or bids 0
\item if \(i\) wins the auction, he pays the amount\linebreak
(price without ring) - (price with ring) \(\geq 0\) \linebreak
to the ring organizer
\item organizer pays \(t_i\) to all ring members, where\linebreak
\(t_i\) = (\(i\)'s expected payment without ring)-(\(i\)'s expected payment with ring)
\end{enumerate}
\end{itemize}
\end{frame}

\begin{frame}[label={sec:orgf04fb56}]{Pre-auction knockout II}
\begin{itemize}
\item is efficient (i.e. bidder with highest valuation wins)
\item makes ring members better off (as they sometimes get a positive payoff even when losing)
\item participation in ring and truthful bidding is an equilibrium
\begin{itemize}
\item implicit assumption: within ring transfers are enforceable
\end{itemize}
\item divert expected revenue from the seller to the ring members
\begin{itemize}
\item effect similar to reducing the numbers of bidders
\pause
\end{itemize}
\item pre-auction knockouts exist(ed) in real-life and were one reason for the development of antitrust legislation
\item FPA seems less vulnerable to collusion (via pre-auction knockout)
\begin{itemize}
\item higher incentives to cheat on the ring if ring bidder bids very low
\item if ring bidder bids not very low, he has to pay his bid
\end{itemize}
\end{itemize}
\end{frame}

\begin{frame}[label={sec:orgc6d206f}]{Asymmetric bidders I}
\begin{itemize}
\item two bidders
\item bidder one: \(v_1\) distributed uniformly on \([0,1]\)
\item bidder two: \(v_2\) distributed uniformly on \([1,2]\)
\item what is outcome of second price auction? what is expected revenue?
\end{itemize}
\vspace*{0.3cm}
\begin{itemize}
\item is the first price auction efficient (i.e. does the bidder with the highest valuation win in equilibrium)?
\item does revenue equivalence hold between FPA and SPA?
\end{itemize}
\end{frame}
\begin{frame}[label={sec:org830a9f3}]{Asymmetric bidders II}
\begin{itemize}
\item one equilibrium of the first price auction:
$$b_1(v_1)=\begin{cases}v_1 & \text{ if }v_1<2/3\\v_1/2+1/3& \text{ else }\end{cases}$$
$$b_2(v_2)=\begin{cases}v_2/2+1/6 & \text{ if }v_2<4/3\\ 5/6& \text{ else }\end{cases}$$
\item check that this is an equilibrium (exercise!)
\item note:
\begin{itemize}
\item bidder 1 has zero probability of winning iff \(v_1\leq 2/3\)
\item bidder 2 has probability 1 of winning iff \(v_2\geq 4/3\)
\item \(b_2(1)<1\), i.e. all types of bidder 2 shade their bids!
\end{itemize}
\item what happens if \(v_1=1\) and \(v_2=1.1\)?
\item is expected revenue lower or higher than in the second price auction?
\end{itemize}
\end{frame}

\begin{frame}[label={sec:org5ba2309}]{Encouraging entry}
\begin{itemize}
\item higher number of bidders increases revenue
\begin{itemize}
\item higher probability of having one/two bidders with high values
\item less bid shading in first price auction
\end{itemize}
\item sometimes argued that first price sealed bid auction encourages entry
\begin{itemize}
\item low value bidders can win against high value bidders (when those sufficiently shade their bids)
\item impossible in an ascending auction (see "asymmetric bidders")
\end{itemize}
\item encouraging even low value bidders to participate in auction can be important
\begin{itemize}
\item design focus on low cost of entry
\begin{itemize}
\item easy access, simple design, information provision, possibly bidding subsidy\ldots{}
\end{itemize}
\end{itemize}
\end{itemize}
\end{frame}
\begin{frame}[label={sec:orgb7c8e44}]{Behavioral considerations}
\begin{itemize}
\item Many of our game-theoretic results have to be taken with a grain of salt.
\item For example, the sealed-bid FPA and the Dutch auction often give different results if played with the same people (and the same \(v_i\) ).
\item Risk-aversion and similar assumptions also mean the RET does not apply.
\item Another consideration:
\begin{itemize}
\item People may derive utility from winning (or resent losing).
\item People may be embarrassed if they win an auction and overpay \(\Ra\) further bid shading.
\end{itemize}
\end{itemize}
\end{frame}

\begin{frame}[label={sec:orgfb3d48f}]{Risk aversion}
\begin{itemize}
\item bidder \(i\) derives utility \(u(v_i-p)\) from the object when winning at price \(p\) (and \(u(-p)\) if he does not win)
\begin{itemize}
\item \(u:\Re\rightarrow\Re\) is strictly increasing and strictly concave
\item assume \(u(0)=0\) and \(u'(0)=1\) for comparison with risk neutrality
\end{itemize}
\item SPA:
\begin{itemize}
\item still optimal to bid valuation (same argument as before combined with \(u\) strictly increasing)
\end{itemize}
\item FPA:
\begin{itemize}
\item expected utility: \(Pr(b_i>b_j)u(v_i-b_i)\)
\item effect of increasing bid
\begin{itemize}
\item increase \(Pr(b_i>b_j)\)
\item decrease \(v_i-b_i\) and therefore \(u(v_i-b_i)\)
\end{itemize}
\item by concavity of \(u\) and \(u'(0)=1\), \(u'(v_i-b_i)<1\) \linebreak \(\Ra\) second effect is weaker than under risk neutrality
\item indeed risk aversion leads typically to higher equilibrium bids than risk neutrality in FPA
\item \(\Ra\) more revenue in FPA than in SPA
\end{itemize}
\end{itemize}
\end{frame}

\begin{frame}[label={sec:org3ead307}]{Example: Embarrassment from overbidding I}
\begin{itemize}
\item sealed-bid FPA in a standard private values model
\item after the auction ends all bids are revealed
\item winning bidder could be said to be “overpaying” by \(b_i -max_{j\neq i} b_ j\)
\item assume that the winning bidder’s utility is
$$v_i - b_i -\gamma (b_ i - max_{j\neq i} b_ j)$$
\item How does this influence each bidder’s expected revenue?
\item How does it influence expected revenue?
\end{itemize}
\end{frame}
\begin{frame}[label={sec:org1a0747a}]{Example: Embarrassment from overbidding II}
$$v_i - b_i -\gamma (b_ i - max_{j\neq i} b_ j)$$
\begin{itemize}
\item “Embarrassment cost” is like an extra payment for the winning bidder.
\item RET applies but "revenue" is overall expenditure by all bidders, including resources “spent” on embarrassment!
\item RET: The expected utility of each bidder is the same as in an FPA (or SPA) without the embarrassment cost.
\item RET: revenue including embarrassment costs is the same as in an FPA (or SPA) without the embarrassment cost.
\item \(\Ra\) expected revenue for the seller is lower due to embarrassment costs
\item Running an ascending auction (where there is no overbidding embarrassment) will increase seller revenue!
\end{itemize}
\end{frame}

\begin{frame}[label={sec:org584aa1e}]{Common-value auctions}
\begin{itemize}
\item So far, we have assumed that each bidder has their own privately known valuation \(v_ i\).
\item alternative assumption: The object has the same value \(v\) for all, but \(v\) is unknown
\item simple case: The wallet auction
\begin{itemize}
\item two bidders; bidder i observes \(t_i\)
\item value of the object is \(v = t_1 + t_2\). i.e. each bidder knows only part of the value (Examples: Oil fields, company takeovers, \ldots{})
\end{itemize}
\item equilibrium in an ascending auction: \linebreak stay in the bidding until \(2t_i\):
\begin{itemize}
\item if other player does so and you stay in the bidding for \(p > 2t_i\) and win, you overpay \(p = 2t_j > t_j + t_i\)
\item if you leave bidding at a lower price you might not win at \(p= 2t_j < t_ i+t_j\)
\item (all other cases: deviation does not change payoff)
\end{itemize}
\end{itemize}
\end{frame}

\begin{frame}[label={sec:orgb518afa}]{Common-value auctions: winner's curse}
\begin{itemize}
\item calculating equilibrium of FPA is somewhat tricky (we will not do it!)
\item with independent private values: bidding your valuation yields zero payoff in FPA
\item what is your expected valuation of the good if\ldots{}
\begin{itemize}
\item \ldots{}you observe \(t_1\) and
\item \ldots{}\(t_2\) is distributed, independently from \(t_1\), uniformly on \([0,1]\)?
\pause
\end{itemize}
\item is your expected payoff also zero if you bid your expected valuation in a common value auction?
\end{itemize}
\end{frame}

\begin{frame}[label={sec:org19b344f}]{Almost common value auctions}
\begin{itemize}
\item Now imagine that the object has a slightly higher value for 1 than for 2, i.e. \(v_ 1 = t_ 1 + t_ 2 + \varepsilon\) and \(v_ 2 = t_ 1 + t_ 2\).
\item Is there an equilibrium in which 2 wins an ascending auction?
\begin{itemize}
\item No: If 2 is willing to pay a price \(p\), then 1 should also be willing to pay that price (and a bit more).
\end{itemize}
\item In equilibrium, 2 never wins the auction!
\item Slight modification completely changes the result \(\Ra\) 2 has no reason to even enter the auction!
\item It can be shown that first price auctions are much less affected (i.e. a change in \(\varepsilon\) in 1’s valuation leads to a similarly-sized change in bidding strategies ⇒ winning probabilities do not change much).
\end{itemize}
\end{frame}

\begin{frame}[label={sec:orge283db5}]{Almost common value auctions: Application}
\begin{itemize}
\item If two bidders compete to take over a company (= common value)\ldots{}
\begin{itemize}
\item \ldots{}but one of them already owns a small stake of the company\ldots{}
\item \ldots{}we have precisely this situation!
\end{itemize}
\item The problem: Takeover battles are by their nature ascending auctions.
\item Example: In Britain in the late 1990s, BSkyB (a TV company) tried to buy a stake in Manchester United.
\begin{itemize}
\item TV licenses (for football broadcasts) are sold in an auction and the money is then distributed to the football clubs.
\item If a TV station owns a bit of a football club, they have an \(\varepsilon\) advantage in bidding.
\item \(\Ra\) In an ascending auction, this can make all the difference!
\end{itemize}
\end{itemize}
\end{frame}


\begin{frame}[label={sec:orgb6a796d}]{What is the right auction format?}
\begin{itemize}
\item It depends!
\item Second-price (and ascending) auctions:
\begin{itemize}
\item appealing properties and are easy to understand (only require dominant strategies!)
\item more robust to distributional assumptions, concerns about secrecy
\item susceptible to collusion, small asymmetries\ldots{}
\end{itemize}
\item First-price auctions:
\begin{itemize}
\item need more sophisticated bidders, sensitive to distributional assumptions etc (require Bayesian NE!)
\item often more robust to collusion and small asymmetries
\end{itemize}
\item Competition between enough bidders can be more important than any theoretical details!
\item Klemperer (2002): “Good auction design is really good undergraduate industrial organization; the two issues that really matter are attracting entry and preventing collusion.”
\end{itemize}
\end{frame}

\section{Reserve prices: market power}
\label{sec:orge09caf0}
\begin{frame}[label={sec:org13c0110}]{Reserve price I}
\begin{itemize}
\item independent private value setting
\item can the seller obtain more expected revenue than in SPA/FPA?
\item consider reserve price \(\bar b\) in a SPA
\begin{itemize}
\item if winning bid is below \(\bar b\) seller keeps the good
\item if second highest bid but not winning bid is below \(\bar b\), the winner has to pay \(\bar b\)
\end{itemize}
\item effectively as if the seller was additional bidder bidding \(\bar b\)
\item still weakly dominant to bid true value
\item can expected revenue be increased by a properly chosen reserve price?
\end{itemize}
\end{frame}

\begin{frame}[label={sec:org6952565}]{Reserve price II (intuition)}
\begin{itemize}
\item start from \(\bar b=0\), i.e. no reserve price
\item what is effect of increasing reserve price slightly to \(\varepsilon>0\)?
\begin{itemize}
\item less revenue if all bidders have value below \(\varepsilon\)
\begin{itemize}
\item probability: \(F(\varepsilon)^I\)
\end{itemize}
\item more revenue if all but one bidder have value below \(\varepsilon\)
\begin{itemize}
\item probability: \(I F(\varepsilon)^{I-1} (1-F(\varepsilon))^1\)
\end{itemize}
\end{itemize}
\item gain is more likely than loss (for \(\varepsilon\) small)
\item (of course size of gain/loss matters as well but the probability effect can be shown to dominate for small \(\varepsilon\))
\end{itemize}
\end{frame}

\begin{frame}[label={sec:org9f4b548}]{Reserve price III (example)}
\begin{itemize}
\item two bidders with valuation uniformly and independently distributed on \([0,1]\)
\item expected revenue without reserve price: \(1/3\)
\item reserve price of \(\bar b\)
\item expected revenue from bidder i:
\end{itemize}
\begin{multline*}
\hspace*{-0.15cm}Pr(v_j<\bar b)*Pr(\bar b<v_i)*\bar b+\int_{\bar b}^1 Pr(\bar b\leq v_j<v_i)\mathbb{E}[v_j|\bar b\leq v_j<v_i]\,dv_i\\
=\bar b*(1-\bar b)*\bar b+\int_{\bar b}^1 (v_i-\bar b)\frac{v_i+\bar b}{2}\, dv_i\\
= (1-\bar b)\bar b^2+\int_{\bar b}^1\frac{v_i^2-\bar b^2}{2}\,dv_i\\
= (1-\bar b)\bar b^2+ \frac{1}{6}-\frac{\bar b^3}{6}-\frac{(1-\bar b)\bar b^2}{2}=\frac{(1-\bar b)\bar b^2}{2}+\frac{1-\bar b^3}{6}\\
=\frac{-4\bar b^3+3\bar b^2+1}{6}
\end{multline*}
\end{frame}
\begin{frame}[label={sec:org4cacff0}]{Reserve price III (example)}
\begin{itemize}
\item expected revenue from bidder i:
$$\frac{-4\bar b^3+3\bar b^2+1}{6}$$
\item maximal for \(\bar b=1/2\) with maximum \(5/24\)
\item maximal revenue: \(2*5/24=5/12>1/3\)
\item note:
\begin{itemize}
\item inefficiency: seller does not value good but good is not sold with probability \(1/4\)
\item similar to monopoly distortion: reduce probability/quantity below efficient level to increase prices and expected profits
\item market incompleteness: monopoly power!
\end{itemize}
\end{itemize}
\end{frame}

\begin{frame}[label={sec:org9ef106f}]{Reserve prices and RET}
\begin{itemize}
\item RET assumption "the object is won by the bidder with the highest valuation and" no longer holds with reserve price
\item however, can be substituted by "the probability of getting a good as a function of type \(P_i(v_i)\) is the same in the compared (auction format, equilibrium) pairs" \linebreak (proof works similarly)
\item hence, a version of RET also holds with reserve prices (though the reserve prices may have to differ between, e.g., SPA and FPA to get the same \(P_i(v_i)\))
\end{itemize}
\end{frame}

\section{Position auctions}
\label{sec:orga32357d}
\begin{frame}[label={sec:orga48f3ef}]{Position auction I}
\begin{itemize}
\item ad position auction (as for Google ads)
\item \(S\) ad slots on a website
\item slot \(s\) has a click through rate of \(x_s>0\) where \(x_1\geq x_2\geq\dots\geq x_S\) (and define \(x_{S+1}=0\) for notational convenience)
\item \(I>S\) bidders with independent private values for an ad slot for a given search term (say "hotel in Cologne")
\item value of slot \(s\) for bidder \(i\) is \(v_i*x_s\)
\begin{itemize}
\item value per click is \(v_i\)
\end{itemize}
\vspace*{0.5cm}
\item new: several slots/prizes of differing prominence
\end{itemize}
\end{frame}
\begin{frame}[label={sec:org050c125}]{Position auction II}
\begin{itemize}
\item consider the following (Vickrey style) auction
\begin{itemize}
\item \emph{s}-highest bidder gets slot \(s\)
\item and pays \(\sum_{t=s+1}^{S+1} (x_{t-1}-x_{t})b_t\) where \(b_t\) is the \(t\) highest bid
\item interpretation: the price of increasing the click through rate from \(x_t\) to \(x_{t-1}\) is the bid of the \emph{t}-highest bidder
\end{itemize}
\item claim: bidding \(v_i\) is weakly dominant strategy
\begin{itemize}
\item bidding above \(v_i\): risk of overpaying, i.e. paying more than \(v_i\) for the last few clicks if someone bids between \(v_i\) and your bid
\item bidding below \(v_i\): risk of missing out, i.e. if the next highest bid is below \(v_i\), a deviation to \(v_i\) would yield additional clicks at a price less than \(v_i\)
\end{itemize}
\end{itemize}
\end{frame}

\begin{frame}[label={sec:orgb65d6b8}]{Position auction III}
\begin{itemize}
\item in practice:
\begin{itemize}
\item click through rates depend on ads (or ad "relevance" to the search term)
\item assign to each advertiser a quality factor \(q_i\)
\item rank advertisers according to \(score_i=b_i*q_i\), i.e. the advertiser with the \(s\) highest score \(b_i*q_i\) gets slot \(s\)
\item prices are directly per actual click
\item used to be:\linebreak price per click of \emph{s}-highest bidder equals \(score_{s+1}/q_s\) (where subscript \emph{s} refers to \emph{s}-highest bidder etc.)
\end{itemize}
\end{itemize}
\end{frame}
\begin{frame}[label={sec:org74d693f}]{Position auction IV}
\begin{itemize}
\item price per click of \emph{s}-highest bidder equals \(score_{s+1}/q_s\)
\item suppose click through rate of ad \(i\) in position \(s\) is \(q_i*x_s\)
\item bidding \(v_i\) is not a dominant strategy:
\begin{itemize}
\item say this leads to slot \emph{s} and payoff \((v_i-b_{s+1}*q_{s+1}/q_i)x_s q_i=x_s(v_iq_i-b_{s+1}q_{s+1})\)
\item deviating to higher bids and getting a better position \(t<s\) is not profitable: deviation payoff \(x_t(v_iq_i-b_tq_t)\) where \(b_t\) is the \emph{t}-highest bid under equilibrium bidding; deviation is negative as \(b_tq_t\) is a higher score than \(b_s=v_i q_i\)
\item deviating to lower bids and getting a worse position \(t>s\):
\begin{itemize}
\item deviation payoff \(x_t(v_iq_i-b_{t+1}q_{t+1})\) may be higher than \(x_s(v_iq_i-b_{s+1}q_{s+1})\) (think, for example, of \(t=s+1\) and \(x_{s}\approx x_{t}\))
\end{itemize}
\item \(\Ra\) bid shading
\end{itemize}
\item \(\Ra\) industry has moved to first price bidding, i.e. price per click is own bid
\end{itemize}
\end{frame}

\section{Markets as double auctions}
\label{sec:org74c28a9}
\begin{frame}[label={sec:org8abcf6c}]{Markets as auctions I}
\begin{itemize}
\item how is a market price determined?
\item stock market:
\begin{itemize}
\item buyers state a bid price
\item sellers state an ask price
\item assume unit demand/supply (otherwise split a buyer who demands \(q\) units up into \(q\) unit demand buyers etc.)
\item stock exchange executes as many trades as possible such that bid price of buyer is above ask price of seller
\begin{itemize}
\item how do you determine which trades to execute?
\end{itemize}
\end{itemize}
\end{itemize}
\end{frame}

\begin{frame}[label={sec:org7ed7e3f}]{Markets as auctions II}
\begin{itemize}
\item how do you determine which trades to execute?
\begin{itemize}
\item order buyers according to bid price:
$$b_1\geq b_2\geq\dots\geq b_I$$
\item order sellers according to ask price:
$$a_1\leq a_2\leq\dots\leq a_J$$
\item execute trades 1 to \emph{k} where \(k\) is the last index where \(b_k\geq a_k\)
\item reminds you of anything?
\end{itemize}
\end{itemize}
\end{frame}

\begin{frame}[label={sec:org1f8ce68}]{Markets as auctions III}
\begin{itemize}
\item what is the trading price?
\begin{itemize}
\item most natural: \((a_k+b_k)/2\)
\item bidding game like first price auction
\item bid shading:
\begin{itemize}
\item bidding true valuation ensures that a buyer (seller) trades if and only if the trading price is below (above) valuation
\item but: positive probability that own bid determines trading price
\item \(\Ra\) buyers (sellers) bid below (above) their valuation to affect trading price favorably in this case
\item if number of buyers and sellers is large, the chance of being buyer/seller \(k\) is small \(\Ra\) hardly any bid shading
\item with bid shading: possible inefficiency as, e.g., trade \(k+1\) may be efficient
\end{itemize}
\item equilibrium strategies can be determined similar to first price auction but somewhat involved
\end{itemize}
\end{itemize}
\end{frame}

\begin{frame}[label={sec:orga967796}]{Markets as auctions IV}
can we do something like a second price auction?
\begin{itemize}
\item hopefully simpler bidding strategies etc.
\item first try:
\begin{itemize}
\item buyers pay \(b_{k+1}\) and sellers receive \(a_{k+1}\)
\item upside: bids of trading players do not affect price
\item \(\Ra\) no incentive for bid shading
\item problem: as \(a_{k+1}>b_{k+1}\), there is a deficit!
\end{itemize}
\end{itemize}
\end{frame}

\begin{frame}[label={sec:orgf137815}]{Markets as auctions V}
can we do something like a second price auction?
\begin{itemize}
\item second try:
\begin{itemize}
\item execute only trades 1 to \(k-1\)
\item buyers pay \(b_k\)
\item sellers receive \(a_k\)
\item now: possibly surplus (say, profit of the stock exchange)
\item weakly dominant strategy to bid true valuation\linebreak for buyer:
\begin{itemize}
\item if trading price (after truthful bid) above valuation no deviation can bring price down \(\Ra\) no profitable deviation
\item if trading price (after truthful bid) below valuation, any deviation would either not affect outcome or lead to not trading \(\Ra\) no profitable deviation
\end{itemize}
\end{itemize}
\item inefficiency: trade \(k\) not executed
\begin{itemize}
\item if many buyers and sellers \(a_k\) and \(b_k\) are close \(\Ra\) inefficiency small
\item for large number of buyers and sellers: no market power \(\Ra\) complete market
\end{itemize}
\end{itemize}
\end{frame}
\end{document}