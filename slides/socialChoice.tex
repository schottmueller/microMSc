% Created 2021-07-19 Mo 14:32
% Intended LaTeX compiler: pdflatex
\documentclass[bigger]{beamer}
\usepackage[utf8]{inputenc}
\usepackage[T1]{fontenc}
\usepackage{graphicx}
\usepackage{grffile}
\usepackage{longtable}
\usepackage{wrapfig}
\usepackage{rotating}
\usepackage[normalem]{ulem}
\usepackage{amsmath}
\usepackage{textcomp}
\usepackage{amssymb}
\usepackage{capt-of}
\usepackage{hyperref}
\mode<beamer>{\useinnertheme{rounded}\usecolortheme{rose}\usecolortheme{dolphin}\setbeamertemplate{navigation symbols}{}\setbeamertemplate{footline}[frame number]{}}
\mode<beamer>{\usepackage{amsmath}\usepackage{ae,aecompl,sgamevar}}
\let\oldframe\frame\renewcommand\frame[1][allowframebreaks]{\oldframe[#1]}
\setbeamertemplate{frametitle continuation}[from second]
\usetheme{default}
\author{Christoph Schottmüller}
\date{}
\title{Social Choice}
\hypersetup{
 pdfauthor={Christoph Schottmüller},
 pdftitle={Social Choice},
 pdfkeywords={},
 pdfsubject={},
 pdfcreator={Emacs 27.2 (Org mode 9.4.4)}, 
 pdflang={English}}
\begin{document}

\maketitle
\begin{frame}{Outline}
\tableofcontents
\end{frame}


\section{Aggregating Preferences}
\label{sec:org3cfc89a}
\begin{frame}[label={sec:org78dc6ae}]{Motivation}
\begin{itemize}
\item people have different preferences
\item how should societal decisions be taken?
\begin{itemize}
\item navigate conflicts of preferences
\item respecting preferences
\end{itemize}
\item Examples:
\begin{itemize}
\item political decisions and elections
\item a group of friends wants to go for drinks: how to aggregate the differing preferences over bars
\item aggregating votes of several judges in sports (boxing, figure skating etc.)
\item (expert) committees
\item a family deciding where to spend the summer holiday
\item \ldots{}
\end{itemize}
\end{itemize}
\end{frame}

\begin{frame}[label={sec:org57b9ac0}]{Social choice theory}
\begin{itemize}
\item make ethical premises explicit
\item derive solutions consistent with these premises
\item normative (!)
\end{itemize}
\end{frame}
\begin{frame}[label={sec:orgb67bb36}]{Example: Majority voting}
\begin{itemize}
\item society (\(N>2\) people) has to choose one of 2 alternatives/candidates (\(x\) and \(y\))
\item assumption for simplicity: everyone has a strict preference over alternatives
\item majority voting:
\begin{itemize}
\item \(x\succeq_S y\) if at least \(N/2\) people prefer \(x\) over \(y\)
\item \(y\succeq_S x\) if at least \(N/2\) people prefer \(y\) over \(x\)
\end{itemize}
\item what normative premises underlie this \emph{social welfare function}?
\end{itemize}
\end{frame}
\begin{frame}[label={sec:org8d863ab}]{Some criteria (for 2 alternatives)}
\begin{block}{Anonymity}
A social choice function is \emph{anonymous} if the names of the agents do not matter, i.e. if a permutation of preferences across agents does not change the social preference.
\end{block}

\begin{block}{Neutrality}
A social welfare function is \emph{neutral} if the names of the alternatives do not matter, i.e. the social preferences are reversed if we reverse the preferences of all agents.
\end{block}

\begin{block}{Positive responsiveness}
A social welfare function is \emph{positively responsive} if the following holds: if one alternative, say \(x\), is weakly socially preferred although \(y\succ_i x\) for some \(i\in\{1,\dots,N\}\), then \(x\) is strictly socially preferred if we change \(i\)'s preferences (without changing anyone else's preferences).
\end{block}
\end{frame}

\begin{frame}[label={sec:org58e187d}]{A first theorem}
\begin{itemize}
\item didn't I claim that social choice starts with premises and then derives solutions?
\end{itemize}

\begin{block}{May's Theorem}
If there are two alternatives, a social welfare function satisfies anonymity, neutrality and positive responsiveness if and only if it is majority voting. 
\end{block}
\begin{block}{Proof sketch ("only if" for even \(N\))}
\begin{itemize}
\item Anonymity: only number of people preferring alternative \(x\) over \(y\) matters for \(\succeq_S\).
\item Neutrality: if \(N/2\) people prefer \(x\) over \(y\), then \(x\approx_S y\).
\item Positive responsiveness: if more than \(N/2\) people prefer \(x\) over \(y\), \(x\succ_S y\) and vice versa.
\end{itemize}
\end{block}
\end{frame}
\begin{frame}[label={sec:orgda31ac1}]{Majority voting with more than 2 alternatives}
\begin{itemize}
\item How to generalize majority voting with more than 2 alternatives?
\end{itemize}
\begin{block}{Definition}
An alternative \(x\) is a \emph{Condorcet winner} if for any other alternative \(y\) a majority prefers \(x\) over \(y\).
\end{block}
\begin{block}{Example}
A group of students want to tell the teacher their preferences over exam forms (open book, closed book, online exam). How to aggregate the preferences?
\begin{center}
\begin{tabular}{l|lll}
 & best & middle & worst\\
\hline
Student 1 & ob & oe & cb\\
Student 2 & oe & cb & ob\\
Student 3 & cb & ob & oe\\
\end{tabular}
\end{center}
Which alternative is Condorcet winner?
\end{block}
\end{frame}


\section{Formal model and criteria}
\label{sec:orgbef9198}

\begin{frame}[label={sec:orgba292e2}]{Model}
\begin{itemize}
\item finite set \(X=\{x_1,x_2,\dots,x_K\}\) of alternatives
\item \(N\geq 2\) agents, each has a complete and transitive preference relation over \(X\)
\end{itemize}
\begin{block}{Social preference relation}
A social preference relation is a complete and transitive preference relation on the set \(X\).
\end{block}
\begin{block}{Social welfare function}
A social welfare function assigns to each profile of preferences \((\succeq_1,\succeq_2,\dots,\succeq_N)\) a social preference relation \(\succeq_S\).
\end{block}
\end{frame}

\begin{frame}[label={sec:org854372b}]{Examples: social welfare function}
Are the following social welfare functions desirable?
\begin{itemize}
\item The preferences of agent 1 are the social preferences: \(\succeq_S (\succeq_1,\succeq_2,\dots,\succeq_N)=\succeq_1\)
\item Fixed social preference relation: \(\succeq_S (\succeq_1,\succeq_2,\dots,\succeq_N)=x_1\succ_S x_2\succ_S x_3\succ_S\dots\succ_S x_K\)
\item Borda Count:
\begin{itemize}
\item turn every agent's preference order into points: the \(k\) most preferred alternative receives \(k\) points
\item for every alternative, sum the points it gets from all agents
\item order alternatives according to points
\end{itemize}
\end{itemize}
\end{frame}
\begin{frame}[label={sec:orga41fd6a}]{Borda and Olympic Ice Skating  competition I}
\begin{itemize}
\item judging in sports is similar to our problem
\begin{itemize}
\item aggregation of several judges' rankings
\end{itemize}
\item final 2002 Olympic figure skating competition
\begin{itemize}
\item Slutskaya is the last skater to perform
\item at that moment: 1. Kwan, 2. Hughes, 3. \ldots{}
\item Slutskaya is doing well but not super and ends up second
\item who came first? who came third?
\end{itemize}
\end{itemize}
\end{frame}

\begin{frame}[label={sec:org1f7819e}]{Borda and Olympic Ice Skating  competition II}
\begin{itemize}
\item say, first rank gives 3 points, second 2 and third 1
\end{itemize}
\begin{center}
\begin{tabular}{l|rrr}
 & Kwan & Hughes & Slutskaya\\
\hline
judge 1 & 2 & 3 & 1\\
judge 2 & 2 & 3 & 1\\
judge 3 & 1 & 2 & 3\\
judge 4 & 1 & 2 & 3\\
judge 5 & 3 & 1 & 2\\
judge 6 & 3 & 1 & 2\\
judge 7 & 3 & 1 & 2\\
\hline
Points &  &  & \\
\end{tabular}
\end{center}
\end{frame}

\begin{frame}[label={sec:org35f6974}]{Minimal (?) normative criteria}
\begin{block}{Weak Pareto principle (unanimity)}
If \(x\succ_i y\) for all \(i=1,2,\dots,N\), then \(x\succ_S y\).
\end{block}
\begin{block}{Non-dictatorship}
There is no individual \(i\) such that \(x\succeq_S y\) if and only if \(x\succeq_i y\). (no matter what other agents preferences are)
\end{block}
\begin{block}{Independence of irrelevant alternatives}
Take two profiles of preferences \((\succeq_1,\succeq_2,\dots,\succeq_N)\) and \((\succeq_1',\succeq_2',\dots,\succeq_N')\). If for every agent \(i\) the ranking of \(x\) and \(y\) is the same under \(\succeq_i\) and \(\succeq_i'\), then the social ranking of \(x\) and \(y\) must be the same under these two preference profiles.\footnote{More formally, let the two preference profiles be such that for all agents $i$ $x\succeq_i y$ if and only if $x\succeq_i' y$. Then $x\succeq_S y$ if and only if $x \succeq_S' y$.}
\end{block}
\end{frame}

\section{Arrow's  impossibility theorem}
\label{sec:org95f26a1}
\begin{frame}[label={sec:orgdc770d1}]{Arrow's impossibility theorem}
\begin{block}{Theorem}
Let there be at least be 3 alternatives in \(X\). There exists no social welfare function that satisfies all 3 criteria (weak Pareto principle, non-dictatorship and independence of irrelevant alternatives). 
\end{block}

Proof is somewhat lengthy (see textbook)
\end{frame}
\begin{frame}[label={sec:org2dc464f}]{Consequences  of Arrow's theorem}
\begin{itemize}
\item no social welfare function satisfies even minimal criteria
\item we have to give up even some of these minimal criteria if we want to proceed!
\item some ways to proceed:
\begin{itemize}
\item pick only one alternative: no complete social ordering necessary
\begin{itemize}
\item leads to similar result
\end{itemize}
\item domain restriction
\begin{itemize}
\item we implicitly assumed that all preference profiles were possible (in the definition "social welfare function")
\item more positive results if we can rule out certain preferences
\end{itemize}
\item cardinal utility
\begin{itemize}
\item we only looked at orderings not at intensity of preference
\item assuming that there is something like intensity of preferences \emph{and this intensity is comparable across agents} helps to aggregate preferences but is a questionable assumption
\end{itemize}
\end{itemize}
\end{itemize}
\end{frame}
\section{Domain restrictions}
\label{sec:orged7b172}
\begin{frame}[label={sec:orgff35be5}]{Domain restriction: Single peaked preferences I}
\begin{itemize}
\item imagine alternatives are ordered on a real line \(x_1<x_2<\dots <x_K\)
\item assumptions:
\begin{itemize}
\item common ordering of alternatives
\item everyone has a most preferred alternative
\item of two "too high" (or "too low") alternatives, an agent prefers the one closer to his most preferred alternative
\item for simplicity: odd number \(N\) of agents
\end{itemize}

\item more precisely:
\begin{itemize}
\item each agent \(i\) has a most preferred alternative \(x^*(i)\in\{x_1,x_2,\dots,x_K\}\)
\item if \(x_k,x_m>x^*(i)\), then \(x_k\succ_i x_m\) if and only if \(x_k<x_m\)
\item if \(x_k,x_m<x^*(i)\), then \(x_k\succ_i x_m\) if and only if \(x_k>x_m\)
\end{itemize}
\item if we represent preferences by utility function, this function is "single peaked"
\end{itemize}
\end{frame}
\begin{frame}[label={sec:org29e2b87}]{Domain restriction: Single peaked preferences II}
\begin{block}{Median agent for single peaked preferences}
An agent \(i\) is a \emph{median agent} if\linebreak (i) there are at least \(N/2\) agents with most preferred alternatives weakly above \(x^*(i)\) and\linebreak (ii) there are at least \(N/2\) agents with most preferred alternatives weakly below \(x^*(i)\).
\end{block}
Note: a median agent always exists.
\end{frame}
\begin{frame}[label={sec:org8fb7ead}]{Domain restriction: Single peaked preferences II}
\begin{block}{Proposition}
Let preferences be single peaked and \(i\) be a median agent, then \(x^*(i)\) is a Condorcet winner.
\end{block}
\begin{block}{Proof}
\begin{itemize}
\item Consider a pairwise majority vote between \(x^*(i)\) and \(x_m>x^*(i)\).

\vspace*{1.5cm}
\item Consider a pairwise majority vote between \(x^*(i)\) and \(x_m<x^*(i)\).

\vspace*{1.5cm}
\end{itemize}
\end{block}
\end{frame}
\begin{frame}[label={sec:org9930e0f}]{Domain restriction: Single peaked preferences III}
\begin{itemize}
\item consider pairwise majority voting between arbitrary alternatives, i.e. say \(x_k\) is socially preferred to \(x_m\) if \(x_k\) wins in a majority vote over \(x_k\) and \(x_m\)
\end{itemize}
\begin{block}{Proposition}
If preferences are single peaked, pairwise majority voting induces a social welfare function.
\end{block}
\begin{block}{Proof}
to show: resulting preferences are complete and transitive

\vspace*{2cm}
\end{block}
\end{frame}

\section{Cardinal utility}
\label{sec:orgb937a82}
\begin{frame}[label={sec:org1bdfdb0}]{Cardinal utility I}
Reminder:
\begin{block}{Representation by a utility function}
A complete preference relation \(\succeq\) over a set \(X\) is represented by the utility function \(u:X\rightarrow\Re\) if and only if
$$x\succeq y \quad\Leftrightarrow\quad u(x)\geq u(y).$$
If \(u\) represents \(\succeq\), then \(\psi(u)\) also represents \(\succeq\) where \(\psi:\Re\rightarrow\Re\) is an arbitrary strictly increasing function.
\end{block}
\end{frame}


\begin{frame}[label={sec:orgabae04e}]{Cardinal utility II}
\begin{itemize}
\item suppose we have 2 agents and \(x\succ_1 y\) while \(y\succ_2 x\)
\item we choose utility functions for the two agents
\begin{itemize}
\item \(u_1(x)=3\), \(u_1(y)=1\)
\item \(u_2(x)=0\), \(u_2(y)=1\)
\end{itemize}
\item which alternative should society prefer?
\end{itemize}
\end{frame}
\begin{frame}[label={sec:org5457fb7}]{Cardinal utility II}
\begin{itemize}
\item if we assign meaning to utility, social welfare function is not invariant to strictly monotone transformations
\item allows to get around Arrow's impossibility theorem
\item problem: choice of specific agent utility functions implicitly makes normative judgments beyond our criteria
\item for now:
\begin{itemize}
\item accept some given utility functions \(u\)
\item let welfare depend on the utilities of the agents and be represented by a function \(W:\Re^N\rightarrow\Re\) that aggregates agent utilities into "welfare"
\begin{itemize}
\item we abuse notation and call \(W\) also "social welfare function"
\end{itemize}
\item what are reasonable choices for \(W\)? what normative judgments are expressed by the choice of \(W\)?
\end{itemize}
\end{itemize}
\end{frame}
\begin{frame}[label={sec:org4673e0b}]{Cardinal utility III}
\begin{block}{Pareto dominance}
Alternative \(x\) is \emph{Pareto dominated} by alternative \(y\) if and only if \(y\succeq_i x\) for all agents \(i=1,..,N\) and \(y\succ_i x\) for at least one agent.
\end{block}
\begin{block}{Pareto efficiency}
An alternative \(x\) is \emph{Pareto efficient} if there is no alternative \(y\) that Pareto dominates \(x\). 
\end{block}
\end{frame}
\begin{frame}[label={sec:org9f2a716}]{Cardinal utility IV}
\begin{block}{Proposition}
If social welfare function \(W\) is strictly increasing, then Pareto dominating alternatives are socially preferred to the alternatives they dominate.
\end{block}
\begin{block}{Proof}
\begin{itemize}
\item let \(W\) be strictly increasing and \(x\) Pareto dominate \(y\)
\end{itemize}

\vspace*{2cm}
\end{block}
\end{frame}

\begin{frame}[label={sec:org073a6de}]{Cardinal utility V: Rawlsian welfare}
$$W_{Rawls}(u_1,\dots,u_N)=\min[u_1,\dots,u_N]$$

\begin{itemize}
\item \(W_{Rawls}\) is strictly increasing \(\Rightarrow\) satisfies Pareto criterion
\item \(W_{Rawls}\) is anonymous
\item \(W_{Rawls}\) is "utility level invariant":
\begin{itemize}
\item social preferences remain the same if we transform all agent's utility using \emph{the same} strictly increasing transformation
\end{itemize}
\item \(W_{Rawls}\) satisfies "Hammond Equity":
\begin{itemize}
\item take two utility vectors \((\bar u_1,\bar u_2,\dots,\bar u_N)\) and \((\hat u_1,\hat u_2,\dots,\hat u_N)\) and suppose \(\bar u_i=\hat u_i\) for all \(i\) except \(j\) and \(k\)
\item suppose further \(\bar u_j<\hat u_j<\hat u_k<\bar u_k\)
\item Hammond equity states that then \(W(\hat u)> W(\bar u)\)
\end{itemize}
\end{itemize}
\end{frame}
\begin{frame}[label={sec:org99671b7}]{Cardinal utility VI: Rawlsian welfare}
\begin{block}{Proposition}
A strictly increasing and continuous social welfare function \(W\) satisfies Hammond equality if and only if it can take the Rawlsian form \(W_{Rawls}(u_1,\dots,u_N)=\min[u_1,\dots,u_N]\).
\end{block}
\begin{itemize}
\item \(\approx\) Rawlsian welfare is equivalent to Pareto criterion  + Hammond equity
\end{itemize}
\begin{block}{Proof}
see Jehle and Reny (2011), section 6.3.1
\end{block}
\end{frame}

\begin{frame}[label={sec:org2af9bd8}]{Cardinal utility VII: Utilitarian welfare}
$$W_{ut}(u_1,\dots,u_N)=\sum_{i=1}^N u_i$$
\begin{itemize}
\item most common form of welfare function (sometimes with individual weights)
\item \(W_{ut}\) is strictly increasing \(\Rightarrow\) satisfies Pareto criterion
\item \(W_{ut}\) is anonymous (not true if weights are used)
\item \(W_{ut}\) is "utility-difference invariant"
\begin{itemize}
\item social preferences are the same if we transform all agents utility using the transformation \(\psi_i(u_i)=a_i+b u_i\)
\end{itemize}
\end{itemize}
\end{frame}

\begin{frame}[label={sec:org656e88b}]{Cardinal utility VIII: Utilitarian welfare}
\begin{block}{Proposition}
A strictly increasing and continuous social welfare function \(W\) satisfies anonymity and utility-difference invariance if and only if it can take the utilitarian form \(W_{ut}=\sum_{i=1}^N u_i\).
\end{block}
\begin{block}{Proof}
see Jehle and Reny (2011), section 6.3.2
\end{block}
\end{frame}
\begin{frame}[label={sec:org166b085}]{Cardinal utility IX: the veil of ignorance I}
\begin{itemize}
\item thought experiment  
\begin{itemize}
\item you will be one of the agents in society
\item you have to decide which alternative to choose
\item you do not know which agent you are going to be
\item some people have argued that whatever a "fair-minded" person would choose in this hypothetical situation is a good societal decision
\end{itemize}
\end{itemize}
\end{frame}
\begin{frame}[label={sec:org4b6c191}]{Cardinal utility X: the veil of ignorance II}
\begin{itemize}
\item Harsanyi:
\begin{itemize}
\item my chance of being agent \(i\) is \(1/N\)
\item my choice should maximizes the expected utility \(\sum_{i=1}^N (1/N) u_i(x)\)
\item \(\rightarrow\) utilitarian welfare
\end{itemize}
\item Rawls:
\begin{itemize}
\item I do not know who I am going to be and there is no basis for assigning probabilities.
\item risk aversion implies maximizing the worst case utility
\item \(\rightarrow\) Rawlsian welfare
\end{itemize}

\item Arrow:
\begin{itemize}
\item Rawls makes a mistake as he assumes not risk aversion but \emph{infinite} risk aversion, i.e. risk aversion does \emph{not} imply maximizing worst case utility.
\end{itemize}
\end{itemize}
\end{frame}

\section{Manipulability}
\label{sec:orgbfd1aab}
\begin{frame}[label={sec:orgae54ad7}]{Manipulability I}
\begin{itemize}
\item so far: preferences of all players are known
\item problem: aggregation
\item what if everyone knows his preferences privately?
\begin{itemize}
\item ask for preferences
\item aggregate
\end{itemize}
\item additional problem: gaming the system by misreporting preferences!
\item result due to Gibbard and Satterthwaite:\linebreak
\emph{If there are at least three alternatives and a social welfare function is (i) Pareto efficient and (ii) creates no gaming possibilities, then it is dictatorial.}
\end{itemize}
\end{frame}

\begin{frame}[label={sec:org153572e}]{Manipulability II}
\begin{itemize}
\item one example for manipulability
\end{itemize}

\begin{block}{Example:   Borda count}
\begin{center}
\begin{tabular}{l|lll}
 & most preferred & middle preferred & least preferred\\
\hline
Agent 1 & x & y & z\\
Agent 2 & y & x & z\\
Agent 3 & y & x & z\\
\hline
Points &  &  & \\
\end{tabular}
\end{center}
Could agent 1 manipulate the social preference relation by misrepresenting his own preferences? Would he want to do so?
\end{block}
\begin{itemize}
\item to discuss such topics properly:\linebreak
extend decision and game theory to incomplete information
\begin{itemize}
\item that's what we will do in the coming weeks!
\end{itemize}
\end{itemize}
\end{frame}


\section{Bibliography}
\label{sec:org5644c75}
\bibliographystyle{chicago}
\bibliography{../../stuff/bibliography/references}
\end{document}